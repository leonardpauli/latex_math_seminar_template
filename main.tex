\newcommand{\SeminarNr}{3}
\newcommand{\Course}{SF1671}
\newcommand{\Author}{MYNAME}
\newcommand{\Date}{MYDATE sep 2020}
\newcommand{\Organization}{KTH}
\newcommand{\ExtraHeaderLeft}{}
%\newcommand{\ExtraHeaderLeft}{, \input{.local.pnr.tex}\!\!} % use to put your personal nr in the header

\documentclass{seminar}
\begin{document}

\section{}

\subsection{Talföljdsbevis med induktion}
\begin{questionquote}
	Talföljden $\{a_n\}_{n=0}^{\infty}$ definieras rekursivt av
	$a_0=0, a_1=1, a_2=4 \text{ och } a_n=4n-6+a_{n-1}-a_{n-2}+a_{n-3},\text{ för }n\ge3.$
	Visa med induktion eller stark induktion att $a_n=n^2$ för alla $n\ge0$.
\end{questionquote}

\dots

\clearpage
\subsection{Kardinalitetsekvivalensbevis i $\R$}
\begin{questionquote}
	Varje reellt tal $x\in[0, 1]$ kan skrivas på decimalform
	$$x=(0, x_1x_2x_3\cdots)_{10}=\sum_{k=1}^{\infty} x_k10^{-k},$$
	där $x_i\in\{0,1,2,\dots,9\}$ för alla i.
	För unikhet förbjuder utvecklingar som slutar med en oändlig följd av 9or.
	Låt $A$ vara mängden av all $x\in[0,1]$ vars decimalutveckling endast innehåller jämna tal.
	Visa att $A$ och $\R$ har samma kardinalitet, d.v.s. $|A|=|\R|$.

	\textit{Bonus.} Visa att det inte finns reella tal
	$0\le a\le b \le 1$ sådant att $(a, b)\subseteq A$.
\end{questionquote}

\dots

\clearpage
\subsection{Lösning av linjär ekvation med modulär aritmetik}
\begin{questionquote}
	Finn alla lösningar till $343x=77$ i $\Z_{805}$.
\end{questionquote}

\dots

\clearpage
\subsection{Rest vid division av stora tal med hjälp av Fermats lilla sats}
\begin{questionquote}
	Bestäm resten av $2771^{3546}$ vid division med $887$.
	\textit{Tips.} Använd Fermats lilla sats.
\end{questionquote}

\dots

\end{document}

% space for notes
% ...
